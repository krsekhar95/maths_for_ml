\documentclass[12 pt]{article}
\usepackage{amsmath}
\begin{document}
\title{\textbf{The Law of Large Numbers and the central limit theoram}}
\author{Kuntla Reddy Sekhar}
\maketitle
\section{Introduction}
The statement of law of large numbers and central limit theoram
are understtod using simulations. The assumptions and the limitations are discussed.
this article concludes with some practical use cases
and applications of central limit theoram.

\section{The Law of Large Numbers}

Let \[X_1,X_2,X_3,....X_n\]  be n independent random variables
drawn from the same probability distribution. These
random variables are generally known as i.i.d's. (Independent identically distribute)
In particular we assume that the mean \(\mu\) and standard deviation
\(\sigma\) of the random variables are same.
we define the mean of the sample as 
\begin{equation}
    \overline{X_n}  = \frac{X_1+X_2+X_3+...+X_n}{n}
    = \frac{1}{n}\sum_{i = 1}^{n}  X_i
\end{equation}
The mean \(\overline{X_n} \) itself is a
random variable. The Law of large numbers and central limit
theoram tells about the value and the 
distribution of the random variable.
\end{document}